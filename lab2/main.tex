\documentclass[a4paper]{article}
\usepackage[utf8]{inputenc}
\usepackage[12pt]{extsizes}
\usepackage{amsmath,amsthm,amssymb}
\usepackage[hidelinks]{hyperref} 
\usepackage[warn]{mathtext}
\usepackage[T1,T2A]{fontenc}
\usepackage[utf8]{inputenc}
\usepackage[english,russian]{babel}
\usepackage{tocloft}
\linespread{1.5}
\usepackage{indentfirst}
\usepackage{setspace}
%\полуторный интервал
\onehalfspacing

\newcommand{\RomanNumeralCaps}[1]
    {\MakeUppercase{\romannumeral #1}}

\usepackage{amssymb}

\usepackage{graphicx, float}
\graphicspath{{pictures/}}
\DeclareGraphicsExtensions{.pdf,.png,.jpg}
\usepackage[left=25mm,right=1cm,
    top=2cm,bottom=20mm,bindingoffset=0cm]{geometry}
\renewcommand{\cftsecleader}{\cftdotfill{\cftdotsep}}

%\addto\captionsrussian{\renewcommand{\contentsname}{СОДЕРЖАНИЕ}}
%\addto\captionsrussian{\renewcommand{\listtablename}{СПИСОК ТАБЛИЦ}}

\usepackage{fancyhdr}
\usepackage[nottoc]{tocbibind}

\fancypagestyle{plain}{
\fancyhf{}
\renewcommand{\headrulewidth}{0pt}
\fancyhead[R]{\thepage}
}

\usepackage{blindtext}
\pagestyle{myheadings}
\usepackage{hyperref}

\begin{document}
\begin{titlepage}
  \begin{center}
    \large
    Санкт-Петербургский политехнический университет Петра Великого
    
    Институт прикладной математики и механики
    
    \textbf{Высшая школа прикладной математики и вычислительной физики}
    \vfill
    \textsc{\textbf{\large{Отчёт по лабораторной работе №2}}}\\[5mm]
     по дисциплине\\ <<Математическая статистика>>\\
\end{center}

\vfill

\begin{tabular}{l p{140} l}
Выполнил студент \\группы 3630102/80401 && Веденичев Дмитрий Александрович \\
\\
Проверил\\Доцент, к.ф.-м.н.& \hspace{0pt} &   Баженов Александр Николаевич \\\\
\end{tabular}

\hfill \break
\hfill \break
\begin{center} Санкт-Петербург \\2021 \end{center}
\thispagestyle{empty}
\end{titlepage}
\newpage
\newpage
\begin{center}
    \setcounter{page}{2}
    \tableofcontents
\end{center}
\newpage
\begin{center}
    \setcounter{page}{3}
    \listoftables
\end{center}

\newpage
\section {Постановка задачи}
\noindent Сгенерировать выборки размером 10, 100 и 1000 элементов.
Для каждой выборки вычислить следующие статистические характеристики положения данных: $\overline{x}, med x, z_R, z_Q, z_{tr}.$ Повторить такие вычисления 1000 раз для каждой выборки и найти среднее характеристик положения и их квадратов:
\begin{equation}
	E(z) = \overline{z}
\end{equation}
Вычислить оценку дисперсии по формуле:
\begin{equation}
	D(z) = \overline{z^2} - \overline{z}^2
\end{equation}
Представить полученные данные в виде таблиц.

\section {Теория}
\subsection{Распределения}
	\begin{itemize}
		\item Нормальное распределение \begin{equation}
										  N(x, 0, 1) = \frac{1}{\sqrt{2\pi}}e^{\frac{-x^2}{2}} \label{norm} 
									   \end{equation}
		\item Распределение Коши \begin{equation}
									C(x, 0, 1) = \frac{1}{\pi}\frac{1}{x^2+1} \label{koshi}
								 \end{equation} 
		\item Распределение Лапласа \begin{equation}
									   L(x, 0, \frac{1}{\sqrt{2}}) = \frac{1}{\sqrt{2}}e^{-\sqrt{2}|x|} \label{laplace} 
									\end{equation}
		\item Распределение Пуассона \begin{equation}
										P(k, 10) = \frac{10^k}{k!}e^{-10}\label{puasson}
									 \end{equation}
		\item Равномерное распределение \begin{equation}
				U(x, -\sqrt{3}, \sqrt{3}) =
				\begin{cases}
					\frac{1}{2\sqrt{3}} &\text{$при |x|\leq \sqrt{3}$}\\
					0 &\text{$при |x|>\sqrt{3}$}
				\end{cases}
				\label{uni} 
			\end{equation}
	\end{itemize}

\subsection{Вариационный ряд}
	\noindent Вариационным рядом называется последовательность элементов выборки, расположенных в неубывающем порядке. Одинаковые элементы повторяются.
	Запись вариационного ряда: $x_{(1)}, x_{(2)}, \ldots, x_{(n)}$.
	Элементы вариационного ряда $x_{(i)} (i = 1, 2, \ldots, n)$ называются порядковыми статистиками.
	
	\subsection{Выборочные числовые характеристики}
	\noindent С помощью выборки образуются её числовые характеристики. Это числовые характеристики дискретной случайной величины $X^{*}$, принимающей выборочные значения $x_{(1)}, x_{(2)}, \ldots, x_{(n)}$.
	
	\subsubsection{Характеристики положения}
	\begin{itemize}
		\item Выборочное среднее \begin{equation}
									 \overline{x} = \frac{1}{n}\sum_{i=1}^{n}{x_i}
								\end{equation}
		\item Выборочная медиана \begin{equation}
								 	med x = \begin{cases}
											 	x_{(l+1)} &\text{$ n=2l+1$}\\
											 	\frac{x_{(l)} + x_{(l+1)}}{2} &\text{$ n=2l$}
								 			\end{cases}
								 \end{equation}
		\item Полусумма экстремальных выборочных элементов \begin{equation}
														       z_R = \frac{x_{(1)} + x_{(n)}}{2}
														   \end{equation}
		\item Полусумма квартилей \newline Выборочная квартиль $z_p$ порядка $p$ определяется формулой \begin{equation}
				 	z_p = \begin{cases}
		             	  	x_{([np]+1)} &\text{$np - $дробное}\\
		      			    x_{(np)}&\text{$np - $целое}
		      			  \end{cases}
				 \end{equation}
				 Полусумма квартилей \begin{equation}
				 					 	z_Q = \frac{z_{1/4} + z_{3/4}}{2}
				 					 \end{equation}
		\item Усечённое среднее\begin{equation}
							   		z_{tr} = \frac{1}{n-2r}\sum_{i=r+1}^{n-r}{x_{(i)}}, r\approx\frac{n}{4}	   	\end{equation}
	\end{itemize}

	\subsubsection{Характеристики рассеяния}
	Выборочная дисперсия
	\begin{equation}
		D = \frac{1}{n}\sum_{i=1}^{n}{(x_i-\overline{x})^2}
	\end{equation}
	
\section {Программная реализация}
\noindent Лабораторная работа выполнена на языке Python вресии 3.7 в среде разработки PyCharm. Использовались дополнительные библиотеки:\\ \newline
1. scipy\newline
2. numpy\newline
3. math\newline
\\
В приложении находится ссылка на GitHub репозиторий с исходныи кодом.

\section {Результаты} 
\subsection{Характеристики положения и рассеяния}
\noindent Как было проведено округление:\\
В оценке $x=\hat{E}$ вариации подлежит разные цифры после точки, в зависимости от распределения. Например в случае распределения Коши\eqref{koshi} вариации подлежат все цифры, так что ни одна не валидна.\\
	\begin{table}[H]
		\centering
		\begin{tabular}[t]{lrrrrr}
			\hline
			Characteristic   &      Mean &    Median &       $z_R$ &      $z_Q$ &      $z_{tr}$ \\
			\hline
			Laplace E(z) 10   &  -0.0009839 & -0.0121517 & -0.0197247 & -0.0076936 & -0.0039645 \\
			Laplace D(z) 10   &  0.0978195 & 0.0683159 & 0.5134668 & 0.4949383 & 0.1625452 \\
			E(z) \pm \sqrt{D(z)} & [ -0.313745 ; & [ -0.2735248 ; & [ -0.7362907 ; & [ -0.7112121 ; & [ -0.4071334 ; \\
			&  0.3117772 ] &  0.2492214 ] &  0.6968413 ] &  0.6958249 ] &  0.3992044 ] \\
			$\hat{E}$(z) & 0.0 & 0.0 & 0.0 & 0.0 & 0.0\\
			\hline
			Laplace E(z) 100  & 0.0034648 & 0.0033212 & 0.0245185 & 0.0027098 & 0.0056071 \\
			Laplace D(z) 100  & 0.0104284 & 0.0062275 & 0.5568239 & 0.4740201 & 0.0208539 \\
			E(z) \pm \sqrt{D(z)} & [ -0.0986547 ; & [ -0.0755933 ; & [ -0.7216878 ; & [ -0.6857814 ; & [ -0.1388017 ; \\
			&  0.1055843 ] &  0.0822357 ] &  0.7707248 ] &  0.691201 ] &  0.1500159 ] \\
			$\hat{E}$(z) & 0.0 & 0.0 & 0.0 & 0.0 & 0.0\\
			\hline
			Laplace E(z) 1000 & -0.0010085 & -0.0015766 & -0.0068676 & -0.0020746 & 0.0001261 \\
			Laplace D(z) 1000 &  0.0011195 & 0.0005789 & 0.4895393 & 0.5319597 & 0.0020667 \\
			E(z) \pm \sqrt{D(z)} & [ -0.0344674 ; & [ -0.0256369 ; & [ -0.7065385 ; & [ -0.7314303 ; & [ -0.0453349 ; \\
			&  0.0324504 ] &  0.0224837 ] &  0.6928033 ] &  0.7272811 ] &  0.0455871 ] \\
			$\hat{E}$(z) & 0.0 & 0.0 & 0.0 & 0.0 & 0.0\\
			\hline
		\end{tabular}
		\caption{Распределение Лапласа \eqref{laplace}}
		\label{tab:normal}
	\end{table}
	
    \begin{table}[H]
		\centering
		\begin{tabular}[t]{lrrrrr}
			\hline
			Characteristic    &      Mean &    Median &       $z_{R}$ &       $z_Q$ &      $z_{tr}$ \\
			\hline
			Uniform E(z) 10   & 0.000659 & -0.0037344 & 0.0087508 & 0.0014502 & -0.0009534 \\   
			Uniform D(z) 10   &  0.1036965 & 0.2388555 & 0.4878255 & 0.4958455 & 0.1611513 \\
			E(z) \pm \sqrt{D(z)} & [ -0.3213604 ; & [ -0.4924629 ; & [ -0.6896943 ; & [ -0.7027128 ; & [ -0.4023899 ; \\
			&  0.3226784 ] &  0.4849941 ] &  0.7071959 ] &  0.7056132 ] &  0.4004831 ] \\
			$\hat{E}$(z) & 0. & 0. & 0. & 0. & 0.\\
			\hline
			Uniform E(z) 100  &  0.003774 & 0.0071305 & 0.0032768 & 0.0214441 & -0.0007186 \\
			Uniform D(z) 100  &  0.009722 & 0.0292367 & 0.4905364 & 0.5358395 & 0.0204128 \\
			E(z) \pm \sqrt{D(z)} & [ -0.0948262 ; & [ -0.1638569 ; & [ -0.6971062 ; & [ -0.7105665 ; & [ -0.143592 ; \\
			&  0.1023742 ] &  0.1781179 ] &  0.7036598 ] &  0.7534547 ] &  0.1421548 ] \\
			$\hat{E}$(z) & 0. & 0. & 0. & 0. & 0.\\
			\hline
			Uniform E(z) 1000 & 0.0004006 & 0.0011303 & 0.0075157 & 0.0083656 & -0.0002115  \\
			Uniform D(z) 1000 &  0.0009215 & 0.0026702 & 0.5022606 & 0.4949788 & 0.0018109 \\
			E(z) \pm \sqrt{D(z)} & [ -0.0299556 ; & [ -0.0505437 ; & [ -0.7011878 ; & [ -0.6951817 ; & [ -0.0427662 ; \\
			&  0.0307568 ] &  0.0528043 ] &  0.7162192 ] &  0.7119129 ] &  0.0423432 ] \\
			$\hat{E}$(z) & 0. & 0. & 0. & 0. & 0.\\
			\hline
		\end{tabular}
		\caption{Равномерное распределение \eqref{uni}}
		\label{tab:uniform}
	\end{table}
	
	\begin{table}[H]
	\centering
		\begin{tabular}[t]{lrrrrr}
			\hline
			Characteristic    &      Mean &    Median &       $z_R$ &       $z_Q$ &      $z_{tr}$ \\
			\hline
			Normal E(z) 10   &  0.0048043 & 0.0035738 & -0.0158131 & 0.029832 & 0.0076417 \\
			Normal  D(z) 10   &  0.093454 & 0.0806334 & 0.4759391 & 0.5494387 & 0.1552688 \\
			E(z) \pm \sqrt{D(z)} & [ -0.3008982 ; & [ -0.2803864 ; & [ -0.7056965 ; & [ -0.7114093 ; & [ -0.3863999 ; \\
			&  0.3105068 ] &  0.287534 ] &  0.6740703 ] &  0.7710733 ] &  0.4016833 ] \\
			$\hat{E}$(z) & 0.0 & 0.0 & 0.0 & 0.0 & 0.0\\
			\hline
			Normal  E(z) 100  &  0.0014876 & -0.004149 & -0.0337122 & 0.0379003 & 0.0061809 \\
			Normal  D(z) 100  & 0.0095173 & 0.0093478 & 0.512915 &  0.4902777 & 0.0180784 \\
			E(z) \pm \sqrt{D(z)} & [ -0.0960691 ; & [ -0.100833 ; & [ -0.749893 ; & [ -0.662298 ; & [ -0.128275 ; \\
			&  0.0990443 ] &  0.092535 ] &  0.6824686 ] &  0.7380986 ] &  0.1406368 ] \\
			$\hat{E}$(z)& 0.0 & 0.0 & 0.0 & 0.0 & 0.0\\
			\hline
			Normal  E(z) 1000 &  7.13e-05 & -0.0002033 & -0.002519 & -0.0178714 & -0.0012642 \\
			Normal  D(z) 1000 &  0.0009921 & 0.0009216 & 0.4910093 & 0.5370306 & 0.0020388 \\
			E(z) \pm \sqrt{D(z)} & [ -0.0314263 ; & [ -0.0305612 ; & [ -0.7032396 ; & [ -0.7506951 ; & [ -0.0464173 ; \\
			&  0.0315689 ] &  0.0301546 ] &  0.6982016 ] &  0.7149523 ] &  0.0438889 ] \\
			$\hat{E}$(z) & 0.0 & 0.0 & 0.0 & 0.0 & 0.0\\
			\hline
		\end{tabular}
		\caption{Нормальное распределение \eqref{norm}}
		\label{tab:laplace}
	\end{table}
	
	\begin{table}[H]
	\centering
		\begin{tabular}[t]{lrrrrr}
			\hline
			Characteristic   &        Mean &    Median &            $z_R$ &       $z_Q$ &      $z_{tr}$ \\
			\hline
			Cauchy E(z) 10   &   -0.9015205 & -0.0102798 & 1.5506815 & 2.0303134 & -1.3998546 \\
			Cauchy D(z) 10   &  887.5172201 & 0.3256897 & 7785.2477732 & 1676.7961675 & 1337.2031198 \\
			E(z) \pm \sqrt{D(z)} & [ -30.6927477 ; & [ -0.5809721 ; & [ -86.6833696 ; & [ -38.9183883 ; & [ -37.9676426 ; \\
			&  28.8897067 ] &  0.5604125 ] &  89.7847326 ] &  42.9790151 ] &  35.1679334 ] \\
			$\hat{E}$(z)& - & 0 & - & - & -\\
			\hline
			Cauchy E(z) 100  &   -0.4475114 & -0.0059119 & -0.4265913 & 2.0329197 & 0.5847918 \\
			Cauchy D(z) 100  & 151.9747356 & 0.025389 & 550.5545438 & 1026.1692437 & 423.7994764  \\
			E(z) \pm \sqrt{D(z)} & [ -12.7753148 ; & [ -0.1652512 ; & [ -23.89049 ; & [ -30.0009568 ; & [ -20.0015988 ; \\
			&  11.880292 ] &  0.1534274 ] &  23.0373074 ] &  34.0667962 ] &  21.1711824 ] \\
			$\hat{E}$(z)& - & 0 & - & - & -\\
			\hline
			Cauchy E(z) 1000 &   0.0604925 & 0.0014351 & 0.1957455 & -0.3087999 & -0.5480516 \\
			Cauchy D(z) 1000 & 1063.4570552 & 0.0023946 & 384.7296859 & 139.1980079 & 3363.2160612 \\
			E(z) \pm \sqrt{D(z)} & [ -32.5501968 ; & [ -0.0474996 ; & [ -19.4187819 ; & [ -12.1070204 ; & [ -58.5412931 ; \\
			&  32.6711818 ] &  0.0503698 ] &  19.8102729 ] &  11.4894206 ] &  57.4451899 ] \\
			$\hat{E}$(z)& - & 0 & - & - & -\\
			\hline
		\end{tabular}
	\caption{Распределение Коши \eqref{koshi}}
	\label{tab:cauchy}
	\end{table}



\begin{table}[H]
		\centering
		\begin{tabular}[t]{lrrrrr}
			\hline
			Characteristic    &      Mean &   Median &       $z_R$ &      $z_Q$ &     $z_{tr}$ \\
			\hline
			Poisson E(z) 10  & 10.0151 & 9.8985 & 9.934 & 10.0265 & 10.0538333     \\
			Poisson D(z) 10   &  0.996602 & 1.3839478 & 5.290144 & 4.8850478 & 1.5409631 \\
			E(z) \pm \sqrt{D(z)} & [ 9.0168004 ; & [ 8.7220869 ; & [ 7.6339687 ; & [ 7.8162856 ; & [ 8.812478 ; \\
			&  11.0133996 ] &  11.0749131 ] &  12.2340313 ] &  12.2367144 ] &  11.2951886 ] \\
			$\hat{E}$(z)& 10^{+1}_{-1} & 10^{+1}_{-1} & 10^{+2}_{-2} & 10^{+2}_{-2} & 10^{+1}_{-1}\\
			\hline
			Poisson E(z) 100  & 9.98215 & 9.82 & 10.113 & 10.094 & 9.9692  \\
			Poisson D(z) 100  &  0.1006763 & 0.1961 & 5.311731 & 4.826664 & 0.1995538 \\
			E(z) \pm \sqrt{D(z)} & [ 9.6648547 ; & [ 9.3771682 ; & [ 7.8082807 ; & [ 7.897033 ; & [ 9.5224855 ; \\
			&  10.2994453 ] &  10.2628318 ] &  12.4177193 ] &  12.290967 ] &  10.4159145 ] \\
			$\hat{E}$(z)& 10^{+0}_{-0} & 10^{+0}_{-1} & 10^{+2}_{-2} & 10^{+2}_{-2} & 10^{+0}_{-0}\\
			\hline
			Poisson E(z) 1000 & 9.995908 & 9.995 & 9.952 & 10.0135 & 9.99356 \\
			Poisson D(z) 1000 &  0.009907 & 0.004475 & 4.938196 & 4.3005678 & 0.0205054 \\
			E(z) \pm \sqrt{D(z)} & [ 9.8963741 ; & [ 9.9281046 ; & [ 7.7297948 ; & [ 7.939719 ; & [ 9.8503629 ; \\
			&  10.0954419 ] &  10.0618954 ] &  12.1742052 ] &  12.087281 ] &  10.1367571 ] \\
			$\hat{E}$(z)& 10^{+0}_{-0} & 10^{+0}_{-0} & 10^{+2}_{-2} & 10^{+2}_{-2} & 10^{+0}_{-0}\\
			\hline
		\end{tabular}
		
		\caption{Распределение Пуассона \eqref{puasson}}
		\label{tab:poisson}
	\end{table}



\section {Обсуждение} 


Из полученных нами данных сильно выделяется распределение  Коши. Так, даже для больших выборок, дисперсия принимает огромные значения. Кроме того, нет какой-то очевидной закономерности между увеличением выборки и изменением значения дисперсии: у mean дисперсия от выборки из 10 к 100 падает, от 100 к 1000 растет, у $z_R$ все время убывает. Данные аномалии являются результами выбросов, которые наблюдались в распределении Коши еще в первой лабораторной.

\section {Приложение}
\noindent Код программы GitHub URL:\\
\newline https://github.com/PopeyeTheSailorsCat/math\_stat\_2021/blob/main/lab2/src/lab2.py

\end{document}